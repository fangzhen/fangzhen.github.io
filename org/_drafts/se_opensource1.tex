% Created 2024-07-03 Wed 17:38
% Intended LaTeX compiler: xelatex
\documentclass{beamer}
\usepackage{graphicx}
\usepackage{longtable}
\usepackage{wrapfig}
\usepackage{rotating}
\usepackage[normalem]{ulem}
\usepackage{amsmath}
\usepackage{amssymb}
\usepackage{capt-of}
\usepackage{hyperref}
\usepackage{ctex}
\usetheme{default}
\usetheme{default}
\date{}
\title{软件工程分享}
\AtBeginSection[]{\begin{frame}<beamer>\frametitle{目录}\tableofcontents[currentsection]\end{frame}}
\hypersetup{
 pdfauthor={},
 pdftitle={软件工程分享},
 pdfkeywords={},
 pdfsubject={},
 pdfcreator={Emacs 29.3 (Org mode 9.6.15)}, 
 pdflang={English}}
\begin{document}

\maketitle

\section{似乎挺简单}
\label{sec:org903aa3f}
\begin{frame}[label={sec:orgba8d112}]{软件开发流程示例}
\begin{center}
\includegraphics[height=0.9\textheight]{../assets/static/opensource/workflow.png}
\end{center}
\end{frame}

\begin{frame}[label={sec:orga4e2a95}]{何谓工程}
\begin{itemize}
\item 注重实践和实际应用
\item 系统化的思维
\begin{itemize}
\item 模型和抽象
\end{itemize}
\item 跨学科协作和组织
\item 项目管理
\begin{itemize}
\item 用户和需求驱动
\item 时间/人员规划
\item 质量控制
\item 规范化流程
\end{itemize}
\end{itemize}
\end{frame}

\begin{frame}[label={sec:orge3ea6ba}]{开源软件中的软件工程实例}
\begin{center}
\begin{tabular}{lll}
\hline
 & Linux Kernel & Openstack\\[0pt]
\hline
模块化设计 & 内核子系统;驱动 & 计算、网络、存储等模块\\[0pt]
\hline
版本控制 & Git & Git\\[0pt]
\hline
代码审查 & 邮件列表 & Gerrit\\[0pt]
\hline
自动化测试 & LKDTM, LTP等 & tox, rally等\\[0pt]
\hline
持续集成 & kernel CI等 & Zuul,Gerrit等\\[0pt]
\hline
文档 & 自动构建 & 自动构建\\[0pt]
\hline
项目管理 & Linux基金会 & Openstack基金会\\[0pt]
\hline
需求管理 & 邮件列表 & Blueprint\\[0pt]
\hline
代码行数 & 3000万 & 主要项目 130万\\[0pt]
\hline
\end{tabular}
\end{center}
\end{frame}

\section{但还是会翻车}
\label{sec:orgdf5eb7d}
\begin{frame}[label={sec:orgbb1d90a}]{案例:VCF软件项目 - 概述}
\begin{itemize}
\item 属于Trilogy项目的一部分,开发一套软件提高信息处理和共享能力,并实现无纸化办公作业。
\item 历时5年开发,总共耗资1.7亿美元
\item 项目需求数度剧变,历任四任CIO
\item 最终项目完全无法使用
\end{itemize}
\end{frame}

\begin{frame}[label={sec:org1fd8f26}]{过程}
\begin{itemize}
\item 立项
\begin{itemize}
\item 时间:2001年6月
\item 目标:升级案件文档管理系统
\item 计划三年,预算1400万美元
\end{itemize}
\item 需求变化
\begin{itemize}
\item 开始一年之内,需求从升级变成重新开发
\item 预算增加,时间延长
\item 利好软件承包商 SAIC 公司
\end{itemize}
\end{itemize}
\end{frame}
\begin{frame}[label={sec:org97a1a3b}]{过程}
\begin{itemize}
\item 混乱:NRC对VCF项目出具的评估报告显示:该项目的开发工作一片混乱,可能从开发伊始就缺乏整体规划。甚至是在项目完工日期之后的几个月,仍然存在下列明显的问题:
\begin{itemize}
\item 探员无法通过该系统将案件资料带到现场进行参考。
\item 系统缺乏最基本的人性化操作特性,连书签和历史记录功能都没有,难以查找有用的资料
\item 系统的排序功能不正常
\item 系统在上线前几乎没有做过测试,上线成败与否完全是随机的
\item 对系统上线可能失败的情况没有做预案,整个系统的上线计划就是一场豪赌。一旦系统上线失败,将彻底失去信息化运作能力。
\end{itemize}
\item 试图挽回
\begin{itemize}
\item 2004年6月,雇佣另一家软件研发公司试图修正项目
\item 该公司出具的报告显示项目已无法挽回
\item 2005年正式宣告失败
\end{itemize}
\end{itemize}
\end{frame}
\begin{frame}[label={sec:orgaae2dc2}]{失败原因}
\begin{itemize}
\item 项目从一开始就缺乏完整的构思,从而导致架构设计的失败
\item 频繁的需求变更
\item 项目管理上频繁往复,导致系统规格混乱
\item 对具体软件开发人员管理过于死板
\item 项目中的很多经理级别管理人员,缺乏基本的计算机科学背景,造成外行领导内行,甚至干扰项目的进行
\item 项目进度严重滞后的情况下,依然不停地添加新的需求
\item 项目需求变更和范围扩大导致的代码膨胀问题
\item 奢望项目能够快速上线投入使用,造成项目无法通过使用磨合提高软件的可用性
\end{itemize}
\end{frame}

\begin{frame}[label={sec:org7691d18}]{启示}
\begin{itemize}
\item 一个软件项目的成败因素是多方面的
\item 明确需求与管理变更
\item 有效的项目管理
\item 技术评估与选择
\item 选择合适的开发模型
\item 供应商管理
\end{itemize}
\end{frame}

\begin{frame}[label={sec:org418e190}]{案例:XZ Utils后门事件 - 概述}
\begin{itemize}
\item XZ Utils项目概述
\begin{itemize}
\item 是.xz文件格式的实现,基于lzma算法,在Linux系统中有着非常广泛的应用
\item 项目维护者Lasse Collin
\end{itemize}
\end{itemize}

后门基本信息:
\begin{itemize}
\item CVE-2024-3094
\item XZ Utils 5.6.0和5.6.1的发布tar包包含后门。这些tar包由Jia Tan创建并签名
\item 攻击者即使没有有效账号,也能通过sshd登陆系统
\item 具体攻击细节仍在调查中
\item 进入了Debian和Fedora的测试版本
\end{itemize}
\end{frame}

\begin{frame}[label={sec:org41e83f7}]{发现过程}
\begin{itemize}
\item 最初发现该问题的是 Postgres 数据库的维护者之一 Andres Freund。
\begin{itemize}
\item 不是专门的安全研究员
\item 来自微软
\end{itemize}
\item 在 Debian 测试版本测试 Postgres 时发现了 openssh-server 的性能问题
\begin{itemize}
\item 登陆时CPU占用高
\item 登陆时间慢了0.5s左右
\end{itemize}
\item 几周的调查之后,基本弄清楚来龙去脉
\begin{itemize}
\item 经过精心设计的供应链投毒
\end{itemize}
\end{itemize}

\url{https://www.openwall.com/lists/oss-security/2024/03/29/4}
\end{frame}

\begin{frame}[label={sec:orgc396379}]{后门植入时间线}
\begin{itemize}
\item 2021 JiaTan Github账号建立,11.16在libarchive提交了第一个PR
\item 之后两年,陆续在xz等项目提交代码,代码质量看起来还比较高
\item 期间有多个不同邮件地址向XZ项目维护者施压,最终JinTan取得了XZ项目的权限
\item 2024年2月23日,Jia Tan将隐藏的后门二进制代码合并到一些二进制测试输入文件中。
\item 2024年2月24日:Jia Tan标记并构建v5.6.0,并发布带有额外恶意build-to-host.m4的xz-5.6.0.tar.gz发行版。
\begin{itemize}
\item 这个m4文件不存在于源代码库中,但打包过程中另外其他合法文件也被添加了,因此它本身并不可疑。但脚本已被改动,添加了后门。
\end{itemize}
\item 5.6.0 版本被发现有ifunc漏洞,该漏洞似乎与攻击无关。
\item 但是发行版另外一些可能修改使得5.6.0中的后门可能失效,JiaTan加速发布了可能包含新后门的5.6.1版本。
\item JiaTan推动新版本进入各大发行版。
\end{itemize}
\end{frame}

\begin{frame}[label={sec:org11734ec}]{}
\begin{center}
\includegraphics[height=0.9\textheight]{../assets/static/opensource/xz-backdoor.jpg}
\end{center}
\end{frame}

\begin{frame}[label={sec:orgcd3d7a1}]{后续}
\begin{itemize}
\item 漏洞被公开后,Lasse和JiaTan的Github账户被暂停。Lasse的账户后来恢复
\item 各大发行版撤下有问题的版本,并发布新版本的包
\item xz项目的官网,邮件列表等去除JiaTan的权限或回退到JiaTan无权限的备份
\item xz源码的Github仓库没有强制push,也就是JiaTan的提交还保留在Git历史中
\begin{itemize}
\item 但是触发build后门的代码没有包含在其中
\end{itemize}
\end{itemize}
\end{frame}

\begin{frame}[label={sec:orgc1141ae}]{启示}
\begin{itemize}
\item 开源软件维护的难度
\item 代码审查的重要性
\item 软件中的小瑕疵带来的影响可能很大
\item 软件工程实践可能在其中发挥作用
\item 开源是提供安全软件的有效途径
\begin{itemize}
\item With enough eyes,all bugs are shallow.(足够多的关注能让所有问题浮现)- Linus
\end{itemize}
\item 没有完美方案
\end{itemize}
\end{frame}

\begin{frame}[label={sec:org5d5cfa5}]{}
\begin{center}
\includegraphics[height=\textheight]{../assets/static/opensource/fragile.png}
\end{center}
\end{frame}

\section{云计算与软件工程}
\label{sec:org9931830}
\begin{frame}[label={sec:orga21baaa}]{云计算为软件工程提供支撑}
\begin{itemize}
\item 开发与部署环境:
\begin{itemize}
\item 云计算提供了灵活的开发和部署环境,开发人员可以随时随地访问和使用计算资源。
\item Online IDE可以更方便标准化配置,提高安全性。
\end{itemize}

\item CICD:
\begin{itemize}
\item 利用云平台提供的能力,开发团队可以更高效地进行持续集成和持续交付(CI/CD)
\end{itemize}

\item 协作和管理:
\begin{itemize}
\item 集成的协作工具和平台,如GitHub、Bitbucket和Jira,支持分布式团队的协作开发。
\end{itemize}

\item 专注核心功能与组件
\begin{itemize}
\item 通过利用各种云服务,降低开发者的心智负担
\end{itemize}
\end{itemize}
\end{frame}

\begin{frame}[label={sec:org7dab50f}]{云计算为软件工程提供支撑}
\begin{itemize}
\item 资源管理:
\begin{itemize}
\item 通过虚拟化技术提供按需资源分配,用户可以根据需要动态调整计算资源的使用。
\end{itemize}

\item 成本效益:
\begin{itemize}
\item 通过按需付费和资源共享,显著降低了硬件和维护成本。
\end{itemize}

\item 安全与合规:
\begin{itemize}
\item 提供了各种安全服务,如身份验证、数据加密和合规性管理,帮助用户保护数据和应用安全。
\end{itemize}
\end{itemize}
\end{frame}

\begin{frame}[label={sec:org539f051}]{云计算本身就是一种软件工程实践}
云计算产品
\begin{itemize}
\item 以开源生态为基础
\begin{itemize}
\item Linux, Kubernetes, Openstack, Ceph, Qemu, Openvswitch etc.
\item 全球协作
\end{itemize}
\item 商业公司提供产品化方案
\end{itemize}

软件工程在云计算软件开发中发挥着关键作用
\begin{itemize}
\item 通过软件工程的系统性思维来解决产品化过程中的问题
\item 着眼于整个产品和方案,而不只是具体的代码实现
\item 可维护性是软件生命周期的一个重要而关键的阶段
\end{itemize}

两者互相促进,提升现代软件开发的效率和效果
\end{frame}

\begin{frame}[label={sec:org092c6da}]{}
\begin{center}
\Huge Thank You!
\end{center}
\end{frame}

\begin{frame}[label={sec:org4b9fd88}]{}
\begin{center}
\Huge Q\&A
\end{center}
\end{frame}
\end{document}
