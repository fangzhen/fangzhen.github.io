% Created 2025-01-24 Fri 19:07
% Intended LaTeX compiler: xelatex
\documentclass[presentation]{beamer}
\usepackage{graphicx}
\usepackage{longtable}
\usepackage{wrapfig}
\usepackage{rotating}
\usepackage[normalem]{ulem}
\usepackage{amsmath}
\usepackage{amssymb}
\usepackage{capt-of}
\usepackage{hyperref}
\usepackage{ctex}
\usepackage{hyperref}
\usepackage{newunicodechar}
\newunicodechar{​}{\ignorespaces} % 忽略零宽空格
\hypersetup{bookmarksdepth = 3}
\usetheme{Hannover}
\usecolortheme{seahorse}
\date{}
\title{Learning To Learn}
\subtitle{学习方法、学习理论调研}
\AtBeginSection[]{\begin{frame}\frametitle{}\setcounter{tocdepth}{2}\tableofcontents[currentsection,sectionstyle=show/shaded,subsectionstyle=show/shaded/hide]\end{frame}}
\hypersetup{
 pdfauthor={},
 pdftitle={Learning To Learn},
 pdfkeywords={},
 pdfsubject={},
 pdfcreator={Emacs 29.4 (Org mode 9.6.15)}, 
 pdflang={English}}
\begin{document}

\maketitle

\section{如何学习}
\label{sec:org3a02deb}
\begin{frame}[label={sec:orgf0851a6}]{《人是如何学习的》}
How People Learn: Brain, Mind, Experience and School

\begin{itemize}
\item 专家与新手的差异:
专家在知识组织、模式识别、问题表征、知识检索和灵活性等方面表现出色,其知识具有系统性、条件化和高效检索等特点,且专家能力的发展需要长期的学习和实践。
\item 学习与迁移(Transfer):
\begin{itemize}
\item 人们必须达到足以支持迁移的初始学习阈值
\item 花费大量时间并不足以确保有效的学习,最重要的是学习时如何利用时间。 理解式学习比仅仅记忆更能激发迁移
\item 在各种不同context下学习到的知识更有可能形成灵活的表示,提取概念的相关特征,更容易迁移
\item 迁移不仅仅在初始学习时一次性发生,而是持续进行的。
\item 既有知识在新情境下也可能有误导性。
\end{itemize}
\item 学习与大脑:
学习改变大脑的物理结构和功能组织,不同的学习经验影响大脑发育,大脑发育具有阶段性和可塑性,且学习与神经活动密切相关。
\end{itemize}
\end{frame}

\begin{frame}[label={sec:org802d6e3}]{费曼学习法}
\begin{itemize}
\item 选择一个概念:选择你想要学习的主题或概念。
\item 教给别人:以一种非常简单的语言将该概念讲解给别人听。你可以假装自己在教一个完全不懂这个概念的人(比如小孩),确保用简单明了的语言进行表达。
\item 找出知识的盲点:在讲解的过程中,你可能会发现自己对某些细节不够了解。记录下你不确定的地方,然后回到相关资料中查找,填补这些知识空白。
\item 简化和总结:一旦你填补了盲点,再次用更简洁的语言来总结这个概念,确保能够以简单的方式解释清楚。
\item 复述并改进:通过反复的练习和复述,逐渐提高自己对这个概念的掌握,最终可以更加流畅和深入地讲解给他人。
\end{itemize}
\end{frame}

\begin{frame}[allowframebreaks,label=]{《刻意练习》}
\begin{itemize}
\item 基本观点
\begin{itemize}
\item 天才源于刻意练习:许多领域的杰出人物都遵循着刻意练习这一相似的成长路径.
\item 强调单纯的长时间练习并不一定能带来卓越成就,关键在于有目的、有计划、有反馈的刻意练习.
\end{itemize}
\item 关键要素
\begin{itemize}
\item 明确目标:细化、可衡量、可操作,例如 “每周提高 10\% 打字准确率”
\item 专注投入:全神贯注,避免分心,引用 “心流” 概念解释专注对练习效果的提升
\item 及时反馈:教练、自我评估或数据监测提供反馈,及时修正错误与偏差
\item 走出舒适区:勇于挑战难度略高于现有水平的任务,促使技能持续进化
\end{itemize}
\end{itemize}

\framebreak
\begin{itemize}
\item 有目的的练习:刻意练习的第一步
\begin{itemize}
\item 天真的练习:一旦某个人的表现达到了“可接受”的水平,并且可以做到自动化,那么,再多“练习”几年,也不会有什么进步。例如开车。
\item 有目的的练习的四个特点:
\begin{itemize}
\item 具有定义明确的目标
\item 专注:把注意力完全集中在你的任务上
\item 包含反馈
\item 需要走出舒适区
\end{itemize}
\item 遇到瓶颈怎么办
\begin{itemize}
\item 寻找正确的方法
\item 动机问题
\end{itemize}
\end{itemize}
\end{itemize}

\framebreak
\begin{itemize}
\item 大脑可塑性
\begin{itemize}
\item 例子
\begin{itemize}
\item 伦敦司机海马体后部更大
\item 盲人阅读盲文时会激活大脑中视觉皮层
\item 训练远视人员看清更小的字 - 眼睛的视觉信号没有变化,但大脑的处理系统更强
\item 音乐家控制手指的区域更大
\end{itemize}
\item 经常性的训练会使大脑中受到训练挑战的区域发生改变
\begin{itemize}
\item 可塑性随年龄增长会变差
\item 不同的能力会相互竞争,此消彼长
\item 需要持续训练才能保持
\end{itemize}
\end{itemize}
\end{itemize}

\framebreak
\begin{itemize}
\item 心理表征
\begin{itemize}
\item 大师与新手记忆棋盘的方式不同
\item 心理表征是一种与我们大脑正在思考的某个物体、某个观点、某些信息或者其他任何事物相对应的心理结构,或具体或抽象
\item 心理表征帮助人们克服短时记忆的局限,迅速地处理大量信息
\item 专家针对本领域的问题创建了高度复杂和精密的表征
\item 心理表征帮助无意识的决策、解释信息、组织信息、制订计划、高效学习
\begin{itemize}
\item 无意识决策:例如攀岩专家不需要有意识的判断每个岩点以何种方式抓握
\end{itemize}
\item 杰出人物运用心理表征来提高技能水平,监测并评估自己的技能水平,在必要时调整心理表征,使之更加有效。
\end{itemize}
\end{itemize}

\framebreak
\begin{itemize}
\item 刻意练习的特点
\begin{itemize}
\item 已经得到合理发展的行业或领域,有一整套套行之有效的训练方法
\item 刻意练习发生在舒适区之外,一般来讲, \alert{过程并不令人愉快}
\item 导师或教练指导,具有明确的目标和计划
\item 包含反馈,以及为应对那些反馈而进行调整的努力
\item 刻意练习既产生有效的心理表征,又依靠有效的心理表征
\end{itemize}
\item 能进行严格意义刻意练习的领域:音乐表演、国际象棋、芭蕾、体操等
\end{itemize}

\framebreak
\begin{itemize}
\item 在其他领域,如何运用刻意练习的原则
\begin{itemize}
\item 确定谁是杰出专家。避免因教育、经验、声望、资历,甚至友善与魅力等产生偏见。
\item 他们何以杰出
\begin{itemize}
\item 揭示杰出人物表现优异的本质原因很难,因为难以深入研究其心理表征,并对其归因
\item 较简单的方式是​\alert{学习他们的训练方法}​,辨别他们与其他人表现的差别
\end{itemize}
\item 最佳方法是找到以为优秀的导师
\end{itemize}
\item 要想成为世界上出类拔萃的顶尖人物,必须经历多年的练习
\begin{itemize}
\item 另一方面,只要以正确的方式训练,人们提高自身绩效和表现的能力将是巨大无比的
\end{itemize}
\end{itemize}

\framebreak
\begin{itemize}
\item 思维转变:
\begin{itemize}
\item 基因差异不重要。从象棋大师、音乐家等各行各业的调查中,没有基因决定的证据。
\item 盲目努力以及只是长时间做某件事,不一定带来提升。
\end{itemize}
\end{itemize}

\framebreak
\begin{itemize}
\item 方法论
\begin{itemize}
\item 从目标角度
\begin{itemize}
\item 意识到问题并找到正确的目标
\item 判断杰出人物做了什么才使他们杰出;找到普通人常犯的错误
\item 创建有反馈的训练工具
\item 聚焦技能(最终表现)而不是知识
\item 最终目的是创建更高质量的心理表征
\end{itemize}
\item 从人的角度
\begin{itemize}
\item 练习过程中保持专注和投入
\item 保持动机\&驱动力
\item 年龄有影响,但成年人仍能学习和改变,即使神经机制可能不同。
\item 我们的身体与大脑在面临挑战时的适应能力最终会胜过任何类型的基因差异。
练习的用途大于天生差异,避免自我实现的预言
\item 精力管理
\end{itemize}
\end{itemize}
\end{itemize}

\framebreak
\begin{itemize}
\item 方法
\begin{itemize}
\item 仿真训练:例如王牌飞行员、放射科医生判断肿瘤
\item 即时反馈:例如外科医生做手术
\item 边干边学
\item 新的物理教学方法
\item 找到好导师
\item 自己设计练习方法:富兰克林提高写作水平的方法
\begin{itemize}
\item 3F方法: focus - feedback - fix it
\end{itemize}
\item 跨越停滞阶段
\begin{itemize}
\item 搞清楚自己的停滞点在什么地方
\end{itemize}
\item 没有一般的“意志力”,但不同人在不同领域的意志力确实有差别。
\item 限制练习时间以保持专注
\begin{itemize}
\item 在较短的时间内投入百分之百的努力来练习,比起在更长时间内只投入70\%的努力来练习,效果更好。
\end{itemize}
\item 你要保持动机,要么强化继续前行的理由,要么弱化停下脚步的理由。
\item 要求自己达到一定目标(跨越停滞阶段)再放弃
\item 保持身体健康,睡眠充足
\item 杰出人物培养路线:
\begin{itemize}
\item 产生兴趣
\item 变得认真:根据他们已经发展出的技能来认同自己,而不再根据其他的兴趣领域(比如选择学校或社交生活)来认同自己。
\item 全力投入
\item 开拓创新:创新者几乎无一例外地在各自的领域或行业中工作了很长时间,已经成为杰出人物,然后再开始开辟新的天地。
\end{itemize}
\item 成为练习人:在一生之中能够通过练习来掌握自己的命运,使得人生充满各种可能。
\end{itemize}
\end{itemize}
\end{frame}

\section{大脑认知机制}
\label{sec:org158d6c1}
\begin{frame}[allowframebreaks=0.9,label=]{《为什么学生不喜欢上学?》}
\begin{itemize}
\item 人们天生好奇,但并非善于思考
\begin{itemize}
\item 大脑不擅长思考,思考是缓慢而且不可靠的。
\item 一个问题的内容,足以激发兴趣,但不足以维持兴趣
\begin{itemize}
\item 我们选择那些 \alert{具有挑战性但似乎可以解决的问题}
\end{itemize}
\item 当你以新的方式组合信息(来自环境的信息和长期记忆)时,思考就会产生。这种结合发生在工作记忆中。
\begin{itemize}
\item 工作记忆有限。随着工作记忆变得拥挤繁多,思考变得越来越困难。
\end{itemize}

\begin{center}
\includegraphics[height=0.4\textheight]{../assets/static/2024/thinking_model.jpg}
\end{center}
\end{itemize}
\end{itemize}


\framebreak
\begin{itemize}
\item 你的思考能力取决于你掌握的事实性知识
\begin{itemize}
\item 错误观点:与其学习事实,不如练习批判性思维,让学生努力评估互联网上的所有信息,而不是试图记住其中的一小部分。
\end{itemize}
\item 背景知识对阅读理解至关重要
\begin{itemize}
\item 提供词汇,帮助理解单个想法的意义
\item 弥合作者留下的逻辑鸿沟,整合信息
\item \alert{允许分块,增加工作记忆的空间}​,更容易将想法联系到一起
\item 指导歧义的解释
\end{itemize}
\item 背景知识对于认知技能的必要性
\begin{itemize}
\item 记忆是首要的认知过程。很多时候,看似思考是在提取记忆。
\item 背景知识可以帮助分块,更多的工作记忆空间对于思考也有作用。
\item 背景知识是广泛的,不止来自书本。
\begin{itemize}
\item “最好的地质学家是见过最多岩石的人” - 赫伯特·里德
\end{itemize}
\end{itemize}
\end{itemize}

\framebreak
\begin{itemize}
\item 事实性知识能提高记忆力
\begin{itemize}
\item 背景知识可以提供记忆的线索
\item 长期记忆中的事实性知识使人们更容易获得更多的事实性知识
\begin{itemize}
\item 意味着差距会越来越大
\end{itemize}
\end{itemize}
\item 与其把知识看作是可能插入思考过程的数据,不如把知识和思考看作是相互交织的。
\end{itemize}


\begin{itemize}
\item 记忆是思考的残留物
\begin{itemize}
\item 情绪、重复等可能都对记忆有帮助
\item 思考内容决定了记忆的内容
\begin{itemize}
\item 老师应该根据希望学生记住的内容设计课程
\end{itemize}
\end{itemize}
\item 如何让学生思考意义
\begin{itemize}
\item 优秀教师具备的品质
\begin{itemize}
\item 能够与学生建立个人联系
\item 以一种有趣且易于理解的方式组织课堂材料
\end{itemize}
\item \alert{\alert{引起注意力 + 引导思考}}
\end{itemize}
\item 人的大脑非常擅长理解和记忆故事
\end{itemize}


\framebreak
\begin{itemize}
\item 为什么抽象的概念难理解
\begin{itemize}
\item 我们通过将新事物与我们已知的事情联系起来去理解它们
\item 能联系起来的多数知识都是具体的,而不是抽象的。例如具体的例子
\item 认知系统在寻找背景知识时,更容易找到浅表知识,而不是深层结构
\begin{itemize}
\item 浅层知识的联系更为直接,例如字面意义上的联系
\item 知识迁移很有难度
\end{itemize}
\item 不同的例子、大量的经验有助于让人看到深层结构
\end{itemize}
\end{itemize}

\framebreak
\begin{itemize}
\item 如果没有长时间的练习,你几乎不可能精通任何脑力工作
\item 练习的好处
\begin{itemize}
\item 强化学习更高阶技能所需的基本技能
\begin{itemize}
\item 让某些过程自动化,释放工作记忆的空间
\end{itemize}
\item 防止遗忘
\item 改善知识迁移
\begin{itemize}
\item 练习使深层结构更加明显。
\item 你可能在第一次知道深层结构时就理解了它,但这并不意味着你会在再次遇到它时就能自动识别出它。
\end{itemize}
\end{itemize}
\end{itemize}
\framebreak
\begin{itemize}
\item 如何练习
\begin{itemize}
\item 练习高频次出现的事情和领域内的关键技能
\item 分散练习:
\begin{itemize}
\item 不同的时间
\item 不同的活动
\end{itemize}
\item 确保练习的多样性
\begin{itemize}
\item 练习表层结构的各种形式有助于理解深层结构。
\end{itemize}
\end{itemize}
\end{itemize}

\framebreak
\begin{itemize}
\item 专家和新手的认知有本质的不同
\begin{itemize}
\item 专家们通过获取广泛的、功能性的 \alert{背景知识} ,并使 \alert{思考过程自动化} ,节省了工作记忆的空间。
\item 专家能够看到和思考深层结构
\end{itemize}
\item 练习是成为专家的关键因素
\begin{itemize}
\item 伟大的科学家有着惊人的毅力,他们的精神疲惫阈值非常高。
\item 10年规则:一个人不能在不到10年的时间内成为任何领域的专家,无论是物理、国际象棋、高尔夫还是数学。
\end{itemize}
\item 新手难以像专家一样学习
\end{itemize}

\framebreak
\begin{itemize}
\item 孩子在思考和学习方面的相似之处多于不同之处
\item 人们有不同的认知风格,如
\begin{itemize}
\item 逻辑型/直觉型:倾向于通过推理学习与倾向于通过洞察力学习
\item 视觉/听觉/动觉型:感知和理解信息的首选方式
\end{itemize}
\item 多元智能:不同的心智活动背后有不同的心智能力,如语言、逻辑、音乐、视觉等
\item 没有科学研究表明根据学习者的认知风格和能力类型来教学能取得更好的效果
\item 遗传和环境都影响智能,但环境影响更大
\item 成长型思维
\begin{itemize}
\item 努力尝试或寻求帮助并不意味着你很笨。
\item 失败并不是因为能力低下,而是因为缺乏经验。
\item 没有必要担心因为失败和暴露你不知道的东西而使你“看起来很笨”​。
\end{itemize}
\end{itemize}
\end{frame}

\begin{frame}[allowframebreaks=0.9,label=]{《重塑杏仁核》}
大脑焦虑的两种途径:
\begin{itemize}
\item 杏仁核通道:感觉信息 - 丘脑 - 杏仁核
\item 大脑皮层通道:(感觉信息 - 丘脑 - )皮层 - 杏仁核
\end{itemize}
激活杏仁核的思维模式
\begin{itemize}
\item 悲观主义
\item 预期:花费大量的时间来思考将来的事
\item 读心术:过于考虑他人的想法,取悦或控制他人
\item 小题大做
\item 完美主义:害怕不完美
\item 过高的目标带来压力
\item 想象中痛苦的画面
\item \alert{仅仅想法(而非现实发生)就会激活杏仁核}
\end{itemize}


\framebreak
识别杏仁核激活的情境 -> 找到激活原因(想法或感官) -> 应对
\begin{itemize}
\item 杏仁核通道 -> 识别触发点,用暴露法缓解
\item 对于只是一些不必要的焦虑想法 -> 意识到只是想法,提出应对想法取而代之
\item 对于确实需要考虑的问题,找到焦虑的问题 -> 提出应对计划 (焦虑具有建设性的事情)
\end{itemize}

安排一个焦虑时间专门用来焦虑,并提醒自己不要在其他时间焦虑
\end{frame}

\section{智能的产生}
\label{sec:org7a235e1}
\begin{frame}[label={sec:org1bfdaff}]{GEB 第11章. 大脑和思维}
\begin{itemize}
\item 把假设性的神经复合体、神经模块统称作符号,表示一个概念,如瀑布,而更复杂的观念,可能需要复杂的符号序列。
\item 符号可以作为其他符号的模板,这一事实给了你的心智以某种相对于现实的独立性,在其中能以你愿意达到的任意精度发生一些不真实的事件,
\item 但作为这一切根源的类符号,是深深植根于现实的。
\begin{itemize}
\item 人类的想象力和讲故事的能力,但是有边界
\end{itemize}
\item 另外两个基本问题:
\begin{itemize}
\item 解释低层次的神经发射通讯如何激活高层次的符号激活通讯
\item 自足地解释高层次的符号激活通讯(建立一个不涉及底层事件的理论)。
\begin{itemize}
\item 人工智能的关键假设,即智能可以实现在其他硬件上。
\end{itemize}
\end{itemize}
\end{itemize}
\end{frame}

\begin{frame}[label={sec:orge75e8dd}]{GEB 第12章 心智和思维}
\begin{itemize}
\item 符号层次上是否可能有大脑之间的同构?不太可能精确同构,否则两个人的思维将不可分别。
但另一方面,某些人的思维比另一些人之间有更多的类似指出,因此,应该有部分同构。
\item 有可能对大脑状态进行组件化的描述。一个大脑状态的组块化描述将由一个带或然性的登记表构成,其中列着一些信念或符号,他们在各式各样的环境中,可能被唤起或触发。
\item 意识存在于哪里?子系统,即符号的集群,与符号没有严格的区别,可以想象为一个‘子脑’,与大脑的其他部分密切联系,又有较强的独立性。
\begin{itemize}
\item 自我子系统与意识:自我子系统有一个非常重要的旁效,即它在下述意义下能扮演灵魂的角色:
\begin{itemize}
\item 在不断地与大脑中其他子系统和符号进行通讯时,它密切注意着哪个符号处于激活状态以及是以什么方式活动。
\item 这就是说,它必定有表示心智活动的符号 —— 或者说,表示符号的符号和表示符号活动的符号。
\end{itemize}
\end{itemize}
\end{itemize}
\end{frame}


\begin{frame}[label={sec:org073bce7}]{复杂系统}
涌现(或称突现、演生、层展)是当许多小的个体相互作用后产生了大的整体,而这个整体展现了构成它的个体所不具备的新特性的现象。
涌现是复杂系统的核心特征。发现复杂系统的涌现规律是复杂性科学的重要目标。
\end{frame}


\section{创造力、抽象与建模}
\label{sec:org351883b}
\begin{frame}[allowframebreaks=0.9,label=]{创造力}
The Art of Doing Science and Engineering: Learning to Learn, Chapter 26: Creativity
\begin{itemize}
\item 创造力(creativity)、原创性(originality)、新颖性(novelty)不同。创造力包含价值的概念
\item 价值可能被埋没。例如不被当时认可的艺术家;不被认可的科学发现(大陆漂移、孟德尔)
\item \alert{创造力似乎是把本来被认为毫不相关的东西"有用地"放在一起}
\begin{itemize}
\item 作者曾把最小二乘法应用于磁学问题
\item 香农把最大似然应用于信息论
\end{itemize}
\item The “set of the mind” at the creative moment enables creativity to be done.
\item 没有证据表明任何提高创造力的方法能够在科学或其他领域显著激发创造力。(包括头脑风暴)
\item 作者认为创造力可以提高,也极为重要。应该从经历过创造体验的人学习。即使有过创造体验,也不一定能说清楚。
\item 历史上做出伟大工作的人,似乎有典型的创造力模式:
\begin{enumerate}
\item 在某种模糊的意义上意识到问题
\item 时间或长或短的问题细化(refinement)。这个阶段不能着急,因为有可能陷入传统方案中。这个阶段需要有激情,有能找到解决方案的信念。
\item 长时间紧张思考可能得到方案,也可能临时放弃。偏执的追求往往不起作用;暂时放弃想法有时似乎对让潜意识找到新方法至关重要。
\item 然后是洞察力时刻。当然,经常找到的方案的错误的。有可能需要修改方案或原始问题。
\end{enumerate}
\item 错误的开始和解法可以让下一个方法更清晰。当卡住时,尝试问自己,如果有解法,它会是什么样子?
\item 有时候,解法从这一切中突然产生。基于这个想法,通常还有很多更进一步的工作要做。
例如清理逻辑,让别人容易理解;换一个视角看待问题和方案,而不是产生洞察的特殊视角。
\item 让要解决的问题长时间地充斥你的潜意识,不要思考别的事情。然后解法可能突然出现。
\begin{itemize}
\item Pasteur: “Luck favors the prepared mind”
\item Newton: "by thinking it constantly"
\end{itemize}
\item 创造力最重要的工具可能是类比。
\begin{itemize}
\item 对知识的广泛了解有帮助
\item 需要对知识的灵活访问,而不只是在有直接线索时。头脑中需要很多hook把知识联系起来。
\begin{itemize}
\item 需要对知识从不同角度审视、深入思考,而不只是记住。
\item 深入到一个领域的基础,在这个过程中不得不从各个角度审视事物。
\item 学习新东西时,想想它的其他可能应用。
\end{itemize}
\item 有价值的类比不一定非常相似,它只需要指明一个方向。
\end{itemize}
\item 通过改变自己以变得更有创造力。了解你自己;从小事做起;依靠自己而不是别人。
\item 放弃一个问题也很重要,否则有可能一直卡住。
\begin{itemize}
\item 例如爱因斯坦对统一理论的研究
\item 年龄对创造力的影响。数学、理论物理领域,大部分开创性工作是年轻时做出的;但文学、音乐等领域并非如此。
\end{itemize}
\end{itemize}
\end{frame}

\begin{frame}[allowframebreaks=0.9,label=]{其他书目}
\setbeamerfont{itemize/enumerate body}{size=\footnotesize}
\setbeamerfont{itemize/enumerate subbody}{size=\scriptsize}

\begin{itemize}
\item 书名: 人是如何学习的: 大脑、心理、经验及学校
\begin{itemize}
\item 原作名: How People Learn
\item 作者/译者: [美] 约翰•D•布兰思福特, 程可拉, 孙亚玲, 王旭卿 /
\item 评价: 7.5 / 564
\item 出版: 华东师范大学出版社 / 2013-1
\item 豆瓣链接: \url{https://book.douban.com/subject/20494282/}
\end{itemize}
\item 书名: 应用学习科学: 心理学大师给教师的建议
\begin{itemize}
\item 原作名: Applying the Science of Learning
\item 作者/译者: [美] 理查德·梅耶, 盛群力, 丁旭, 钟丽佳 /
\item 评价: 9.1 / 128
\item 出版: 中国轻工业出版社 / 2016-10
\item 豆瓣链接: \url{https://book.douban.com/subject/27082580/}
\end{itemize}
\item 书名: 剑桥学习科学手册: 
\begin{itemize}
\item 原作名: The Cambridge Handbook of the Learning Sciences
\item 作者/译者: [美]R.基思·索耶, 徐晓东 /
\item 评价: 8.8 / 188
\item 出版: 教育科学出版社 / 2010
\item 豆瓣链接: \url{https://book.douban.com/subject/4836530/}
\end{itemize}
\item 书名: 为什么学生不喜欢上学?: 
\begin{itemize}
\item 原作名: Why Don't Students Like School?: A Cognitive Scientist Answers Questions About How the Mind Works and What It Means for the Classroom
\item 作者/译者: 赵萌, 朱永新(审校) /
\item 评价: 9.2 / 5062
\item 出版: 江苏教育出版社 / 2010-5
\item 豆瓣链接: \url{https://book.douban.com/subject/4864832/}
\end{itemize}
\item 书名: 学会如何学习: 
\begin{itemize}
\item 原作名: Learning How to Learn: How to Succeed in School Without Spending All Your Time Studying
\item 作者/译者: [美]芭芭拉·奥克利(Barbara Oakley), [美]特伦斯·谢诺夫斯基(Terrence Sejnowski), [英]阿利斯泰尔·麦康维尔(Alistair McConville), 汪幼枫 /
\item 评价: 8.4 / 588
\item 出版: 机械工业出版社 / 2020-1
\item 豆瓣链接: \url{https://book.douban.com/subject/34923186/}
\end{itemize}
\item 书名: 如何学习: 
\begin{itemize}
\item 原作名: How We Learn: The Surprising Truth About When, Where, and Why It Happens
\item 作者/译者: [美] Benedict Carey / 玉冰
\item 评价: 8.0 / 2928
\item 出版: 浙江人民出版社 / 2017-7-15
\item 豆瓣链接: \url{https://book.douban.com/subject/27081766/}
\end{itemize}
\item 书名: 创造力: 心流与创新心理学
\begin{itemize}
\item 原作名: 米哈里•希斯赞特米哈伊
\item 作者/译者: Mihaly Csikszentmihalyi, 黄珏苹 /
\item 评价: 7.9 / 792
\item 出版: 浙江人民出版社 / 2015-1-1
\item 豆瓣链接: \url{https://book.douban.com/subject/26285299/}
\end{itemize}
\item 书名: 伟大创意的诞生: 创新自然史
\begin{itemize}
\item 原作名: Where Good Ideas Come From: The Natural History of Innovation
\item 作者/译者: [美] 史蒂文·约翰逊, 盛杨燕 /
\item 评价: 7.7 / 607
\item 出版: 浙江人民出版社 / 2014-8-1
\item 豆瓣链接: \url{https://book.douban.com/subject/25958751/}
\end{itemize}
\item 书名: 科技群星闪耀时: 15个创新传奇
\begin{itemize}
\item 原作名: Idea Makers: Personal Perspectives on the Lives \& Ideas of Some Notable People
\item 作者/译者: [美] 斯蒂芬•沃尔弗拉姆(Stephen Wolfram), 应俊耀, 蔚怡 /
\item 评价: 7.6 / 25
\item 出版: 人民邮电出版社 / 2024-8
\item 豆瓣链接: \url{https://book.douban.com/subject/36991789/}
\end{itemize}
\item 书名: The Art of Doing Science and Engineering: Learning to Learn
\begin{itemize}
\item 原作名:
\item 作者/译者: Richard W. Hamming, Bret Victor /
\item 评价: 9.8 / 24
\item 出版: Stripe Press Books / 2020-5-26
\item 豆瓣链接: \url{https://book.douban.com/subject/35084167/}
\end{itemize}
\item 书名: 涌现: 从混沌到有序
\begin{itemize}
\item 原作名:
\item 作者/译者: [美] 约翰·霍兰 / 陈禹
\item 评价: 8.7 / 111
\item 出版: 上海科学技术出版社 / 2001-11
\item 豆瓣链接: \url{https://book.douban.com/subject/1543713/}
\end{itemize}
\item 书名: A New Kind of Science: 
\begin{itemize}
\item 原作名:
\item 作者/译者:  /
\item 评价: 8.5 / 158
\item 出版: 158人评价 / 2002-5
\item 豆瓣链接: \url{https://book.douban.com/subject/1468622/}
\end{itemize}
\item 书名: 弹性: 在极速变化的世界中灵活思考
\begin{itemize}
\item 原作名: Elastic: Flexible Thinking in a Time of Change
\item 作者/译者: [美] 列纳德·蒙洛迪诺
\item 评价: 6.9 / 480
\item 出版: 中信出版集团 / 2019-8-1
\item 豆瓣链接: \url{https://book.douban.com/subject/34625758/}
\end{itemize}
\item 书名: 刻意练习: 如何从新手到大师
\begin{itemize}
\item 原作名: Peak: Secrets from the New Science of Expertise
\item 作者/译者: 罗伯特·普尔(Robert Pool), 王正林 /
\item 评价: 7.8 / 24555
\item 出版: 机械工业出版社 / 2016-11-6
\item 豆瓣链接: \url{https://book.douban.com/subject/26895993/}
\end{itemize}
\item 书名: 颠覆性创新: 
\begin{itemize}
\item 原作名:
\item 作者/译者: 克莱顿·克里斯坦森, 崔传刚 /
\item 评价: 8.4 / 92
\item 出版: 中信出版社 / 2019-11
\item 豆瓣链接: \url{https://book.douban.com/subject/34897714/}
\end{itemize}
\item 书名: 创新者的窘境: 
\begin{itemize}
\item 原作名: The Innovator's Dilemma: When New Technologies Cause Great Firms to Fail
\item 作者/译者: 胡建桥 /
\item 评价: 8.6 / 4003
\item 出版: 中信出版社 / 2010-6
\item 豆瓣链接: \url{https://book.douban.com/subject/4243770/}
\end{itemize}
\item 书名: 人类的认知: 思维的信息加工理论
\begin{itemize}
\item 原作名:
\item 作者/译者: 荆其诚, 张厚粲 /
\item 评价: 9.0 / 101
\item 出版: 科学出版社 / 1986-11
\item 豆瓣链接: \url{https://book.douban.com/subject/26587908/}
\end{itemize}
\item 书名: 创造性: 人类创新的科学
\begin{itemize}
\item 原作名: Explaining Creativity: The Science of Human Innovation
\item 作者/译者: [美] R. Keith Sawyer, 师保国 /
\item 评价: 9.5 / 32
\item 出版: 华东师范大学出版社 / 2013-9
\item 豆瓣链接: \url{https://book.douban.com/subject/25743277/}
\end{itemize}
\item 书名: 科学革命的结构: 
\begin{itemize}
\item 原作名: The Structure of Scientific Revolutions
\item 作者/译者: 托马斯·库恩, 张卜天 /
\item 评价: 9.1 / 664
\item 出版: 北京大学出版社 / 2022-7-21
\item 豆瓣链接: \url{https://book.douban.com/subject/35951747/}
\end{itemize}
\item 书名: 狭义与广义相对论浅说: 
\begin{itemize}
\item 原作名: Relatioity, The Special And The General Theory(A Popular Exposition)
\item 作者/译者: 杨润殷 /
\item 评价: 9.3 / 1235
\item 出版: 北京大学出版社 / 2006-1
\item 豆瓣链接: \url{https://book.douban.com/subject/1707050/}
\end{itemize}
\end{itemize}
\end{frame}
\end{document}
