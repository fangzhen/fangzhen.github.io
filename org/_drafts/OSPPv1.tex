% Created 2025-03-13 Thu 14:05
% Intended LaTeX compiler: xelatex
\documentclass[presentation]{beamer}
\usepackage{graphicx}
\usepackage{longtable}
\usepackage{wrapfig}
\usepackage{rotating}
\usepackage[normalem]{ulem}
\usepackage{capt-of}
\usepackage{hyperref}
\usepackage{ctex}
\usepackage{hyperref}
\usepackage{newunicodechar}
\newunicodechar{​}{\ignorespaces} % 忽略零宽空格
\hypersetup{bookmarksdepth = 3}
\usetheme{Hannover}
\usecolortheme{seahorse}
\date{}
\title{读书笔记}
\subtitle{操作系统:原理与实践 - 第一卷 内核与进程}
\AtBeginSection[]{\begin{frame}\frametitle{}\setcounter{tocdepth}{2}\tableofcontents[currentsection,sectionstyle=show/shaded,subsectionstyle=show/shaded/hide]\end{frame}}
\hypersetup{
 pdfauthor={},
 pdftitle={读书笔记},
 pdfkeywords={},
 pdfsubject={},
 pdfcreator={Emacs 30.1 (Org mode 9.7.11)}, 
 pdflang={English}}
\begin{document}

\maketitle
\section{引言}
\label{sec:org1862e91}
\begin{frame}[label={sec:orgd9f1a55}]{引言}
考虑一个Web server,操作系统的部分工作是使得开发类似web server这样的应用更加容易。从Web server的角度,操作系统可能要提供一些支持,比如:
\begin{itemize}
\item Web server需要访问文件或调用外部程序计算并获取结果 - 操作系统怎么让不同的应用互相通信?
\item 客户端可以并发访问,为获得更好的体验,server同时处理多个客户端的请求 - 操作系统如何支持应用同时做多件事情
\item 随着业务增长,web server可能要部署在其他硬件上 - 操作系统如何屏蔽硬件差异,以避免重写web server?
\end{itemize}
\end{frame}
\begin{frame}[label={sec:orga286c9a}]{操作系统定义}
操作系统是一个软件层,管理计算机资源,为用户及其应用程序提供服务。
\end{frame}
\begin{frame}[label={sec:org47243d4}]{操作系统角色}
操作系统作为硬件与应用的中间层,需要扮演三个角色:
\begin{itemize}
\item 裁判(Referee):资源共享。应用共享硬件资源,需要操作系统决定资源何时分配给哪个应用。
\item 幻术师(Illusionist):提供一个物理硬件的抽象层,使得应用仿佛拥有完整的、资源不受限的计算机从而简化设计。例如应用不必关注物理内存分布、不必关注其他应用的资源竞争。
\item 粘合剂(Glue):为应用程序提供公共服务或组件,例如文件系统、统一的用户界面观感、IO。
\end{itemize}
\end{frame}
\begin{frame}[label={sec:orgddf1700}]{其他系统的类似角色}
除了操作系统,很多其他系统也有类似的设计挑战,比如,云计算、浏览器、数据库系统、互联网。以云计算为例:
\begin{itemize}
\item Referee:资源在云上不同应用之间如何分配?
\item Illusionist:云的资源一直在变化,应该如何向应用开发者隔离硬件变化?
\item Glue:云上服务通常跨主机运行。云软件需要提供那些服务来协调不同的服务以及共享数据?
\end{itemize}
\end{frame}
\begin{frame}[label={sec:org0a059ef}]{操作系统评估标准:}
\begin{itemize}
\item 可靠性和可用性:系统是否按照预期运行及其运行时间比例。
\item 安全性:是否能够防止恶意攻击及保护用户数据隐私。
\item 可移植性:系统是否易于适配新的硬件。
\item 性能:包括响应时间、吞吐量以及可预测性。
\item 流行度:应用程序、硬件、用户生态。
\end{itemize}
\end{frame}
\begin{frame}[label={sec:orgf3da491}]{操作系统:过去现在与未来}
\begin{itemize}
\item 早期操作系统:操作系统主要是通过提供一些公共服务(如标准IO程序)来减少编程错误。
\item 批处理操作系统
\begin{itemize}
\item 依次对每个任务加载、执行、卸载的步骤。在一个任务执行时,利用DMA,可以并行进行IO操作。
\item 多任务操作系统:允许多个任务并发执行,当一个任务需要IO给出CPU时,其他任务可以执行,提高了处理器利用率。也带来了更高的隔离需求。
\item 虚拟机最早用来解决批处理系统难以Debug的问题。
\end{itemize}
\item 分时操作系统:支持交互式使用,而不只是批处理。
\item 现代操作系统:随着计算机技术的发展,各种现代操作系统运行在不同的硬件上,包括个人电脑、智能手机、嵌入式设备、服务器等。
\item 未来操作系统展望:
\begin{itemize}
\item 安全性和可靠性的更高要求
\item 操作系统需要适应底层硬件发展。数据中心向大规模、异构方向发展;无处不在的便携硬件,如各种可穿戴设备。
\end{itemize}
\end{itemize}
\end{frame}
\section{内核抽象}
\label{sec:orgbd1f7bb}

\begin{frame}[label={sec:orgbaafc44}]{概述}
\begin{itemize}
\item 操作系统的一个核心角色是保护——把misbehaving的程序和用户隔离开来,以避免影响其他应用或操作系统本身。
\item 保护是实现某些操作系统其他目标的基础。如安全性、可靠性、隐私、公平资源分配。
\item \alert{\alert{进程是内核提供的受保护执行的抽象}}
\item 这种设计模式——可扩展的程序运行第三方脚本——在其他领域也有很多,比如浏览器。
\end{itemize}
\end{frame}
\begin{frame}[label={sec:org96be17f}]{进程抽象}
\begin{itemize}
\item 进程是程序运行实例,
\begin{itemize}
\item 执行指令
\begin{itemize}
\item program counter
\item 不同类型的指令:算术、分支、访问IO等
\end{itemize}
\item 访问内存
\begin{itemize}
\item 代码
\item 数据
\item 堆
\item 栈
\end{itemize}
\end{itemize}
\item 操作系统通过进程控制块(PCB)管理和追踪进程
\item 本章只讨论简单的单线程进程
\end{itemize}
\end{frame}
\begin{frame}[label={sec:orgcd3c170}]{进程保护}
操作系统如何防止进程危害其他进程或者操作系统本身?
\begin{itemize}
\item 不考虑性能,一个假想的方案:由内核来执行用户进程的指令。
\begin{itemize}
\item 对于用户态,在执行每个指令前检查其权限,如
\begin{itemize}
\item 是否会跳转到其他进程内存?
\item 是否访问其他进程的数据?
\end{itemize}
\end{itemize}
\item 如果想要用户态代码直接在处理器上运行,硬件至少需要提供以下支持:
\begin{itemize}
\item 特权指令:只能在内核态执行,不能在用户态执行;例如
\begin{itemize}
\item 用户态代码不能修改自己能访问的内存范围
\item 用户态代码不能关闭中断
\end{itemize}
\item 内存保护
\begin{itemize}
\item 用户进程只能读写自己的内存,而不能访问其他进程或kernel的内存
\item 内存保护的一种简单方案:每个进程通过base和bound两个寄存器指定合法的内存地址。
\begin{itemize}
\item 相比虚拟地址方案,缺少很多特性。
\end{itemize}
\end{itemize}
\item 时钟中断:允许内核周期性重获处理器的控制权,才有可能进行多任务调度。
\end{itemize}
\end{itemize}
\end{frame}
\begin{frame}[label={sec:org33f0830}]{模式切换}
\begin{itemize}
\item 用户态到内核态
\begin{itemize}
\item 外部中断
\item 处理器异常
\item 系统调用
\end{itemize}
\item 内核态到用户态
\begin{itemize}
\item 新建进程
\item 切换进程
\item 从中断、异常、系统调用中返回
\item Upcall(内核向用户态发起的调用): 如linux下的信号
\end{itemize}
\end{itemize}
\end{frame}
\begin{frame}[label={sec:orga8fdd9c}]{实现安全的模式切换}
\begin{itemize}
\item context switch 的实现至少应该满足:
\begin{itemize}
\item 受限地进入内核:用户进程不能任意访问内核内存,硬件必须确保进入内核的入口点是由内核设置的入口点。
\item 处理器状态的原子切换:例如不能发生PC切换到内核态代码,但内存保护状态还没切换到内核态。
\item 中断处理对被中断的进程透明。
\end{itemize}
\item 需要软硬件共同实现,相关机制包括:
\begin{itemize}
\item 中断向量
\item Interrupt stack/Kernel stack:处理器和内核配合,在中断或系统调用时切换栈
\item 中断屏蔽:有些内核代码执行过程不能被中断,需要屏蔽中断
\item 保存/恢复被中断的现场,主要是寄存器。
\end{itemize}
\end{itemize}
\end{frame}
\begin{frame}[label={sec:org4531dd6}]{具体实现}
\begin{itemize}
\item x86中断处理的模式切换
\item 系统调用的实现
\item 创建新进程
\item 实现upcall
\end{itemize}
\end{frame}
\begin{frame}[label={sec:orgb51eb89}]{Case Study}
\begin{itemize}
\item 从BIOS到bootloader到Kernel启动过程
\item 虚拟机
\end{itemize}
\end{frame}
\section{编程接口}
\label{sec:org87a8eef}
\begin{frame}[label={sec:org86cbb9d}]{概述 - 如何使用进程抽象}
\begin{itemize}
\item 操作系统给应用提供的功能,包括
\begin{itemize}
\item 进程管理
\item IO
\item 线程管理
\item 内存管理
\item 存储与文件系统
\item 网络
\item 图形接口与窗口管理
\item 认证与安全
\end{itemize}
\item 这些也是本章及本书接下来的核心: \alert{从进程抽象出发,探讨操作系统设计的原则与实践}
\end{itemize}
\end{frame}
\begin{frame}[label={sec:org4bb5f11}]{概述 - 操作系统的组织}
\begin{itemize}
\item 这些功能在哪实现(kernel还是用户库)
\begin{itemize}
\item 用户态程序:如用户登陆管理
\item 用户态链接库:如窗口组件
\item kernel中,通过syscall访问:如进程调度、内存管理
\item kernel启动的内核态进程:如很多系统中的窗口管理
\end{itemize}
\item 权衡:
\begin{itemize}
\item 灵活性
\item 可靠性
\item 性能
\item 安全性
\end{itemize}
\end{itemize}
\end{frame}
\begin{frame}[label={sec:org618af29}]{进程管理}
\begin{itemize}
\item 早期批处理系统中,完全由kernel来管理进程
\begin{itemize}
\item 用户提交任务,操作系统在合适的时间初始化并启动进程
\end{itemize}
\item 现代操作系统中,允许用户程序来创建和管理它们自己的进程
\begin{itemize}
\item 如Shell、Web服务器、窗口管理器、浏览器、数据库、编译器等等
\item 每个用户程序管理按自己的逻辑管理进程,例如
\begin{itemize}
\item Web服务器主进程监听请求,子进程处理请求
\item Shell运行用户输入的命令,还可以提供一些任务管理功能
\end{itemize}
\end{itemize}
\item 内核提供基本的进程管理相关的系统调用
\begin{itemize}
\item Windows: `CreateProcess`
\item UNIX: `fork` \& `exec`
\end{itemize}
\end{itemize}
\end{frame}
\begin{frame}[label={sec:org6a13c86}]{输入/输出}
\begin{itemize}
\item IO设备多种多样:例如键盘、鼠标、显示器、磁盘、以太网卡等
\item 选择1:给每种设备提供对应的编程接口
\begin{itemize}
\item 问题:一旦有新的设备类型,就需要提供新的系统调用
\end{itemize}
\item UNIX的一个主要创新是通过单一的通用接口统一所有设备输入输出。UNIX IO接口的基本思想如下:
\begin{itemize}
\item 统一:所有文件操作、设备IO、进程间通信都使用同一套系统调用:`open`, `close`, `read`, `write`。
\item 使用前先open
\item 面向字节
\item 内核缓冲读写:块设备或流设备的读写都需要经过内核,使得读写接口保持一致。使读写操作与具体设备解耦。
\item 显式close
\end{itemize}
\item 要实现进程间通信,还需要几个概念:
\begin{itemize}
\item 管道:管道是内核缓冲,一个进程写入管道,另一个进程读取,从而实现通信。
\item IO重定向:标准输入输出+IO重定向可以使得进程不必关心输入/输出对应的实际文件。
\item IO多路复用:当server/client模式下,server通过多个pipe与多个client相连时。
\end{itemize}
\end{itemize}
\end{frame}
\begin{frame}[label={sec:org5808af7}]{案例研究}
\begin{itemize}
\item Shell实现
\begin{itemize}
\item 上面的IO模型使得实现以下行为更为简单:
\begin{itemize}
\item 程序可以是包含命令的文件(shell脚本):shell解释器不需要区分键盘键入的命令还是从文件读入的命令。
\item 程序可以使用文件作为标准输入/输出:标准输入/输出+IO重定向
\item 通过管道连接两个程序,实现生产者/消费者模型
\end{itemize}
\end{itemize}
\item 进程间通信:几个普遍使用的进程间通信方式都与IO模型紧密相关:
\begin{itemize}
\item 生产者-消费者:管道
\item 客户端-服务器:客户端与服务器都是通过IO 接口与kernel交互来读写。
\item 文件系统
\end{itemize}
\end{itemize}
\end{frame}
\begin{frame}[label={sec:orga245bd2}]{操作系统结构}
操作系统要提供很多功能,有些需要在内核中实现,有些可以放在用户态实现(比如上面的shell)。
操作系统不同模块之间的交互也有很多,例如
\begin{itemize}
\item 很多模块依赖同步原语
\item 虚拟内存管理模块依赖硬件的地址翻译
\item 文件系统和内存管理共享物理内存;它们可能都依赖磁盘驱动
\end{itemize}

因此操作系统设计需要权衡,更多在在kernel中实现
\begin{itemize}
\item 有助于提高性能,以及模块间更好地集成
\item 但也会更不灵活、难以修改。
\end{itemize}
\end{frame}
\begin{frame}[label={sec:org65c2af2}]{操作系统结构}
\begin{itemize}
\item 宏内核
\begin{itemize}
\item 广泛使用的商用内核大部分选择宏内核,包括Windows, Linux, MacOS
\item 为了提高可移植性,大部分现代操作系统的硬件管理都包含两个特性:
\begin{itemize}
\item 硬件抽象层
\item 驱动动态加载
\end{itemize}
\end{itemize}
\item 微内核
\begin{itemize}
\item 在用户空间运行更多的操作系统服务
\item 例如窗口管理器在大多数操作系统中是用户态实现的
\end{itemize}
\item 对于操作系统开发者,微内核更容易模块化以及Deug
\item 但对于用户程序来说,微内核的好处并不显著(可能有潜在的可靠性提高),但很可能带来性能损失
\end{itemize}
\end{frame}
\begin{frame}[label={sec:org8915268}]{未来方向}
例如
\begin{itemize}
\item 应用成为一个自成体系的小型操作系统
\begin{itemize}
\item 当前的操作系统的设计中没有考虑这种情况。比如系统调用。
\end{itemize}
\item 资源分配决策对应用程序显式可见:传统上操作系统透明地进行资源分配决策,以提高整体性能。
\begin{itemize}
\item 这种透明性意味着应用程序不知道资源使用情况,也无法做出优化。
\end{itemize}
\end{itemize}
\end{frame}
\end{document}
